\documentclass[a4paper,11pt,twocolumn]{article}
\usepackage[latin1]{inputenc}
\usepackage[english]{babel}
\usepackage{amsmath}
\usepackage{amsfonts}
\usepackage{amssymb}

\usepackage{titling}
\usepackage{nomencl}
\usepackage{siunitx}
\usepackage[style=ieee,backend=bibtex]{biblatex}
\usepackage[font={small}]{caption}

\usepackage{graphicx}
\usepackage{color}

\usepackage{booktabs}
\usepackage{threeparttable}
\usepackage{fancyhdr}
\usepackage{float}

\usepackage{varioref}
\usepackage{xspace}
\usepackage[activate={true,nocompatibility},final,tracking=true,kerning=true,spacing=nonfrench,factor=1100,stretch=10,shrink=10]{microtype}
% activate={true,nocompatibility} - activate protrusion and expansion final -
% enable microtype; use "draft" to disable tracking=true, kerning=true,
% spacing=true - activate these techniques factor=1100 - add 10% to the
% protrusion amount (default is 1000) stretch=10, shrink=10 - reduce
% stretchability/shrinkability (default is 20/20) Reduce tracking around small
% caps to 40%
\SetTracking{encoding={*}, shape=sc}{40}

% Document info.
\author{Z0966990}
\title{M23 Lab Report}
\date{\today}

% Path to images.
\graphicspath{{img/}}

% Setup nomenclature.
\newlength{\nomtitlesep}
\setlength{\nomtitlesep}{\nomitemsep}
\setlength{\nomitemsep}{-0\parsep}
\renewcommand\nomgroup[1]{
    \ifthenelse{\equal{#1}{A}}{
        \itemsep\nomtitlesep
        \item[\textbf{Acronyms}]
        \itemsep\nomitemsep}{
    \ifthenelse{\equal{#1}{B}}{
        \itemsep\nomtitlesep
        \item[\textbf{Theory}]
        \itemsep\nomitemsep}{
    \ifthenelse{\equal{#1}{C}}{
        \itemsep\nomtitlesep
        \item[\textbf{Experiment}]
        \itemsep\nomitemsep}{
    \ifthenelse{\equal{#1}{D}}{
        \itemsep\nomtitlesep
        \item[\textbf{Simulation}]
        \itemsep\nomitemsep}{
}}}}}
\newcommand{\nomindex}[1]{
    \hspace*{\fill}
    \makebox[5em][l]{#1}
}
\makenomenclature

% Units.
\newcommand{\GPa}{\si{\giga\pascal}\xspace}
\newcommand{\MPa}{\si{\mega\pascal}\xspace}
\newcommand{\mm}{\si{\milli\meter}\xspace}
\newcommand{\micro}{\si{\micro}\xspace}

% Setup bibiliography.
\addbibresource{m23_lab_report.bib}

% Header and footer.
\pagestyle{fancy}
\fancyhf{}
\lhead{\thetitle}
\rhead{\theauthor}
\cfoot{\thepage}
\renewcommand{\headrulewidth}{0pt}
\renewcommand{\footrulewidth}{0pt}

\begin{document}
    
% Title page.
\begin{titlepage}
    \centering
    \vspace*{\fill}
    \includegraphics[width=0.5\textwidth]{Durham.png}\\
    \vspace*{\fill}
    \LARGE\thetitle\\
    \large\theauthor\\
    \large L2 Mechanical Engineering\\
    \large\thedate\\
    \vspace*{\fill}
\end{titlepage}

% Main matter.
\renewcommand{\abstractname}{\large Abstract}
\twocolumn[
\begin{@twocolumnfalse}
    \begin{abstract}
        In this report it is detailed how experimentation and computational
        simulation were used to investigate how stress and strain concentrations
        in plates vary around features.
    \end{abstract}
\end{@twocolumnfalse}
\vspace{\parsep}
]

% Acronyms
\nomenclature[A0]{CSA}{Computational stress analysis}
\nomenclature[A1]{FEA}{Finite element analysis}

% Theory
\nomenclature[B00]{$E$}{Young's modulus of elasticity
    \hspace{\fill}\GPa}
\nomenclature[B01]{$S$}{Far-field stress
    \hspace{\fill}\MPa}
\nomenclature[B02]{$k_\epsilon$}{Strain concentration
    \hspace{\fill}---}
\nomenclature[B03]{$k_\sigma$}{Stress concentration
    \hspace{\fill}---}
\nomenclature[B04]{$\gamma_{uv}$}{Shear strain in $uv$-plane
    \hspace{\fill}\micro}
\nomenclature[B05]{$\epsilon_{nom}$}{Nominal strain
    \hspace{\fill}\micro}
\nomenclature[B06]{$\epsilon_{uu}$}{Normal strain in $u$-direction
    \hspace{\fill}\micro}
\nomenclature[B07]{$\nu$}{Poisson's ratio
    \hspace{\fill}---}
\nomenclature[B08]{$\sigma_{nom}$}{Nominal stress
    \hspace{\fill}\MPa}
\nomenclature[B09]{$\sigma_{uu}$}{Normal stress in $u$-direction
    \hspace{\fill}\MPa}
\nomenclature[B10]{$\sigma_{uv}$}{Shear stress in $uv$-plane
    \hspace{\fill}\MPa}

% Experiment
\nomenclature[C0]{$k_{\epsilon i}$}{Strain concentration at gauge $i$
    \hspace{\fill}---}
\nomenclature[C1]{$k_{\sigma i}$}{Stress concentration at gauge $i$
    \hspace{\fill}---}
\nomenclature[C2]{$\epsilon_i$}{Strain at gauge $i$
    \hspace{\fill}\micro}
\nomenclature[C3]{$\sigma_i$}{Stress at gauge $i$
    \hspace{\fill}\MPa}
\nomenclature[C4]{$\sigma_{max}$}{Maximum plate stress
    \hspace{\fill}\MPa}

% Simulation
\nomenclature[D0]{$H$}{Height of plate
    \hspace{\fill}\mm}
\nomenclature[D1]{$L$}{Length of plate
    \hspace{\fill}\mm}
\nomenclature[D2]{$d$}{Height of plate feature
    \hspace{\fill}\mm}
\nomenclature[D3]{$r$}{Radius of plate feature
    \hspace{\fill}\mm}

\printnomenclature

\section{Introduction}
\section{Theory}
\section{Method}
\subsection{Experiment}

\begin{figure}[h]
    \def\svgwidth{0.48\textwidth}
    \input{img/experiment_dims.pdf_tex}
    \caption{Dimensions of plate used in experiment.}
    \label{fig:experiment_dims}
\end{figure}

\begin{figure}[h]
    \def\svgwidth{0.48\textwidth}
    \input{img/experiment_gauges.pdf_tex}
    \caption{Positions of strain gauges on plate used in experiment.}
    \label{fig:experiment_gauges}
\end{figure}

\subsection{Simulation}

\begin{figure}[h]
    \def\svgwidth{0.48\textwidth}
    \input{img/simulation_dims.pdf_tex}
    \caption{Dimensions of plate used in simulation.}
    \label{fig:simulation_dims}
\end{figure}

In practice, the plate detailed in Figure~\ref{fig:simulation_dims} was modelled
with the feature height $d$ kept constant at 100~\mm. To investigate the effects
of varying the ratio $H/d$, the simulations were run with $H$ at 110~\mm,
130~\mm and 200~\mm. For each $H$ value, $r/d$ was also varied by running the
simulations with six different feature radiuses in the range 5~\mm--10~\mm.

$L$ was chosen to be ten times greater than $H$, sufficiently long such that the
plate behaved as if the far-field stress was applied an infinite distance away
from the shoulder fillets.

The nominal far-field stress $S$ was applied to the plate using a traction of
magnitude 100~\MPa. The magnitude of this load was not important as stress
concentrations are normalised by the chosen far-field stress.

\section{Discussion}
\section{Conclusions}

% References.
\printbibliography

\end{document}
\label{fig:experiment_dims}