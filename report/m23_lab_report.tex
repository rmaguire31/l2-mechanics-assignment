\documentclass[a4paper,11pt,twocolumn]{article}
\usepackage[latin1]{inputenc}
\usepackage[english]{babel}
\usepackage{amsmath}
\usepackage{amsfonts}
\usepackage{amssymb}
\usepackage{mathtools}

\usepackage{titling}
\usepackage{nomencl}
\usepackage{siunitx}
\usepackage[style=ieee,backend=bibtex]{biblatex}
\usepackage[font={small}]{caption}

\usepackage{graphicx}
\usepackage{color}

\usepackage{booktabs}
\usepackage{threeparttable}
\usepackage{calc}
\usepackage{fancyhdr}
\usepackage{float}

\usepackage{textcomp}
\usepackage{varioref}
\usepackage{xspace}
\usepackage[activate={true,nocompatibility},final,tracking=true,kerning=true,spacing=nonfrench,factor=1100,stretch=10,shrink=10]{microtype}
% activate={true,nocompatibility} - activate protrusion and expansion final -
% enable microtype; use "draft" to disable tracking=true, kerning=true,
% spacing=true - activate these techniques factor=1100 - add 10% to the
% protrusion amount (default is 1000) stretch=10, shrink=10 - reduce
% stretchability/shrinkability (default is 20/20) Reduce tracking around small
% caps to 40%
\SetTracking{encoding={*}, shape=sc}{40}

% Document info.
\author{Z0966990}
\title{M23 Lab Report}
\date{\today}

% Path to images.
\graphicspath{{img/}}

% Setup nomenclature.
\newlength{\nomtitlesep}
\setlength{\nomtitlesep}{\nomitemsep}
\setlength{\nomitemsep}{-0.5\parsep}
\renewcommand\nomgroup[1]{
    \ifthenelse{\equal{#1}{A}}{
        \itemsep\nomtitlesep
        \item[\textbf{Acronyms}]
        \itemsep\nomitemsep}{
    \ifthenelse{\equal{#1}{B}}{
        \itemsep\nomtitlesep
        \item[\textbf{Background}]
        \itemsep\nomitemsep}{
    \ifthenelse{\equal{#1}{C}}{
        \itemsep\nomtitlesep
        \item[\textbf{Experiment}]
        \itemsep\nomitemsep}{
    %else
        \itemsep\nomtitlesep
        \item[\textbf{\textcolor{red}{Undefined}}]
        \itemsep\nomitemsep
}}}}
\makenomenclature

% Setup bibiliography.
\addbibresource{m23_lab_report.bib}

% Header and footer.
\pagestyle{fancy}
\fancyhf{}
\lhead{\thetitle}
\rhead{\theauthor}
\cfoot{\thepage}
\renewcommand{\headrulewidth}{0pt}
\renewcommand{\footrulewidth}{0pt}

% Macros.
\newcommand{\CSA}{{\sc CSA}\xspace}
\newcommand{\FEA}{{\sc FEA}\xspace}
\newcommand{\SE}{{\sc SE}\xspace}

\newcommand{\GPa}{\si{\giga\pascal}\xspace}
\newcommand{\MPa}{\si{\mega\pascal}\xspace}
\newcommand{\mm}{\si{\milli\meter}\xspace}

\begin{document}
    
% Title page.
\begin{titlepage}
    \centering
    \vspace*{\fill}
    \includegraphics[width=0.5\textwidth]{Durham.png}\\
    \vspace*{\fill}
    \LARGE\thetitle\\
    \large\theauthor\\
    \large L2 Mechanical Engineering\\
    \large\thedate\\
    \vspace*{\fill}
\end{titlepage}

% Main matter.
\renewcommand{\abstractname}{\large Abstract}
\twocolumn[
\begin{@twocolumnfalse}
    \begin{abstract}
        In this report it is detailed how experimentation, computational
        simulation and analytical approaches were used to investigate how stress
        and strain concentrations in plates vary around features.
    \end{abstract}
\end{@twocolumnfalse}
\vspace{\parsep}
]

% Acronyms
\nomenclature[A0]{\CSA}{Computational stress analysis}
\nomenclature[A1]{\FEA}{Finite element analysis}
\nomenclature[A2]{\SE}{Standard error}
\nomenclature[AA]{~}{\vspace*{-\baselineskip}}

% Theory
\nomenclature[B00]{$E$}{Young's modulus of elasticity
    \hspace*{\fill}\GPa}
\nomenclature[B01]{$H$}{Height of plate
    \hspace*{\fill}\mm}
\nomenclature[B02]{$K_\epsilon$}{True strain concentration factor
    \hspace*{\fill}---}
% \nomenclature[B03]{$K_{\epsilon_t}$}{Theoretical strain concentration factor\\
%     \hspace*{\fill}---}
\nomenclature[B04]{$K_\sigma$}{True stress concentration factor
    \hspace*{\fill}---}
\nomenclature[B05]{$K_{\sigma_t}$}{Theoretical stress concentration factor\\
    \hspace*{\fill}---}
\nomenclature[B06]{$L$}{Length of plate
    \hspace*{\fill}\mm}
\nomenclature[B07]{$S$}{Far-field stress
    \hspace*{\fill}\MPa}
\nomenclature[B08]{$d$}{Height/diameter of plate feature
    \hspace*{\fill}\mm}
\nomenclature[B09]{$r$}{Radius of plate feature
    \hspace*{\fill}\mm}
\nomenclature[B10]{$\gamma_{uv}$}{Shear strain in $uv$-plane
    \hspace*{\fill}---}
\nomenclature[B11]{$\epsilon_{nom}$}{Nominal strain
    \hspace*{\fill}---}
\nomenclature[B12]{$\epsilon_{ww}$}{Normal strain in $w$-direction
    \hspace*{\fill}---}
\nomenclature[B13]{$\nu$}{Poisson's ratio
    \hspace*{\fill}---}
\nomenclature[B14]{$\sigma_{max}$}{Maximum plate stress
    \hspace*{\fill}\MPa}
\nomenclature[B15]{$\sigma_{nom}$}{Nominal stress
    \hspace*{\fill}\MPa}
\nomenclature[B16]{$\sigma_{ww}$}{Normal stress in $w$-direction
    \hspace*{\fill}\MPa}
\nomenclature[B17]{$\sigma_{uv}$}{Shear stress in $uv$-plane
%     \hspace*{\fill}\MPa}
\nomenclature[BB]{}{\vspace{-\baselineskip}}

% Experiment
\nomenclature[C0]{$K_{\epsilon_i}$}{Strain concentration factor at gauge $i$\\
    \hspace*{\fill}---}
\nomenclature[C1]{$K_{\sigma_i}$}{Stress concentration factor at gauge $i$\\
    \hspace*{\fill}---}
\nomenclature[C2]{$\epsilon_i$}{Strain at gauge $i$
    \hspace*{\fill}---}
\nomenclature[C3]{$\sigma_i$}{Stress at gauge $i$
    \hspace*{\fill}\MPa}
\nomenclature[CC]{}{\vspace{-\baselineskip}}

\printnomenclature

\section{Introduction}

When designing structural components, engineers typically look to the most
stressed point in that component to determine and specify failure criteria and
factors of safety.

In simple systems, engineers can use intuition to identify candidates for
possible points of failure. In more complex systems, such as components
with smoothly varying features, more rigorous approaches to stress analysis 
are required. Experimentation, simulation and theory can all identify the most 
severely stressed point in a component.

In this report, finite width plates with either circular holes or shoulder
fillets were analysed. Circular holes are a common feature present in many
parts, allowing for fixtures such as screws and bolts to be used. Shoulder
fillets are introduced to reduce the peak stress experienced by components with
sharp internal corners.

\section{Background}

\subsection{Stress Measurements}

Plane stress problems are defined by their zero out of plane stress. The most 
common plane stress situations involve stresses resolved on the surface of 
plates. Lautrup~\cite[p.~129]{lautrup2004physics} gives a set of equations for
Hooke's law in plane stress for an isotropic material which can be represented
in matrix form as follows:
\begin{equation} \label{eq:hooke-plane-stress}
    \begin{pmatrix*}[l]
        \sigma_{xx} \\
        \sigma_{yy} \\
        \sigma_{xy}
    \end{pmatrix*}
    = \frac{E}{1-\nu^2}
    \begin{pmatrix*}[c]
        1   & \nu & 0 \\
        \nu & 1   & 0 \\
        0   & 0   & \frac{1+\nu}{2}
    \end{pmatrix*}
    \begin{pmatrix*}[l]
        \epsilon_{xx} \\
        \epsilon_{yy} \\
        \gamma_{xy}
    \end{pmatrix*}
\end{equation}

Equation~\eqref{eq:hooke-plane-stress} can be used to determine the stress at a 
point on the surface of a the plate using measurements of strain from strain 
gauges. A strain gauge rosette consists of three gauges positioned as detailed 
in Figure~\vref{fig:rosette}. Hibbeler~\cite[p.~512]{hibbeler2017mechanics} 
offers the following transformations to find $\epsilon_{xx}$, $\epsilon_{yy}$ 
and $\epsilon_{xy}$:
\begin{align}
    \label{eq:rosette-xx}
    \epsilon_{xx} &= \epsilon_a \\
    \label{eq:rosette-yy}
    \epsilon_{yy} &= \epsilon_c \\
    \label{eq:rosette-xy}
    \gamma_{xy}   &= 2\epsilon_b - (\epsilon_a + \epsilon_b)
\end{align}

\begin{figure}[h]
    \centering
    \def\svgwidth{0.48\textwidth}
    \input{img/rosette.pdf_tex}
    \caption{45\si{\degree} strain gauge rosette.}
    \label{fig:rosette}
\end{figure}

On the plate boundary, the only non-zero stress component is tangential to
the plate boundary. The relationship between the tangential stress and strain 
can be described using of Hooke's law under axial load, given by 
Hibbeler~\cite[p.~88]{hibbeler2017mechanics}:
\begin{equation} \label{eq:hooke-axial}
    \sigma_{xx} = E\epsilon_{xx}
\end{equation}
where $x$ is the axial or tangential direction.

In the axial load case, shear strain is also zero. The Poisson
effect can be used to determine Poisson's ratio if the axial and
transverse strains are known. The following expression was also given by
Hibbler~\cite[p.~106]{hibbeler2017mechanics}:
\begin{equation} \label{eq:poisson}
    \nu = -\frac{\epsilon_{xx}}{\epsilon_{yy}} 
        = -\frac{\epsilon_{xx}}{\epsilon_{zz}}
\end{equation}

\subsection{Stress Concentrations}

The stress concentration factor is a dimensionless quantity which describes how
much greater the maximum principle stress in the feature is than the 
nominal---or  average---stress in the part. Similarly, there also exists a 
strain concentration factor, which is a measure of relative deformation.
According by Pilkey~\cite[p.~4]{pilkey2008peterson}:
\begin{align}
    \label{eq:stress-conc}
    K_\sigma &= \frac{\sigma_{max}}{\sigma_{nom}} \\
    \label{eq:strain-conc}
    K_\epsilon &= \frac{\epsilon_{max}}{\epsilon_{nom}}
\end{align}

For simple features---or stress raisers---such as the 
circular hole in Figure~\vref{fig:hole-contour}, the \emph{theoretical} stress
concentration factor $K_{\sigma_t}$ can be determined analytically as described
by Pilkey~\cite[p.~181]{pilkey2008peterson}:
\begin{equation} \label{eq:inf-hole-conc}
    K_{\sigma_t} = 3
\end{equation}

\begin{figure}[h]
    \centering
    \def\svgwidth{0.48\textwidth}
    \input{img/hole_conc.pdf_tex}
    \caption{Local stress concentration around hole.}
    \label{fig:hole-contour}
\end{figure}

Pilkey also offers equations to determine the local stress concentrations at
any point in the plate; used to produce the contour plot in 
Figure~\ref{fig:hole-contour}. These reveal most stressed---red---points lie
at the top and bottom of the hole.

However, when the plate has a finite width as illustrated in 
Figure~\vref{fig:hole-dims}, the problem becomes much harder to solve
analytically. Pilkey~\cite[p.~183]{pilkey2008peterson} offered a more
generalised stress concentration factor expression, obtained by fitting
empirical data:
\begin{equation} \label{eq:hole-conc}
    K_{\sigma_t} = 2 + \left(1 - \frac{d}{H}\right)^3
\end{equation}

\begin{figure}[h]
    \centering
    \def\svgwidth{0.48\textwidth}
    \input{img/hole_dims.pdf_tex}
    \caption{Finite width plate with circular hole.}
    \label{fig:hole-dims}
\end{figure}

In the limit as $H\rightarrow\infty$,
$\eqref{eq:hole-conc}\rightarrow\eqref{eq:inf-hole-conc}$.

For the shoulder fillets detailed in Figure~\vref{fig:shoulder-dims}, Kumagai et
al.~\cite{kumagai1968stress} suggested an expression for stress concentration 
factor derived empirically:
\begin{equation} \label{eq:shoulder-conc}
    K_{\sigma_t} = \left(\frac{H/d-1}{2(2.8H/d-2)}\frac{d}{r}\right)^{0.65}
\end{equation}

\begin{figure}[h]
    \centering
    \def\svgwidth{0.48\textwidth}
    \input{img/shoulder_dims.pdf_tex}
    \caption{Finite width plate with shoulder fillets.}
    \label{fig:shoulder-dims}
\end{figure}

However, to determine the location of the most stressed points, engineers can
either measure or simulate the stress concentration factors throughout the
plate. \FEA is modern computational method which divides the complex part into
simpler elements using a mesh, resolving the stresses in each element. Prior to
the development of \FEA, photoelasticity was the most common approach, chosen by
Kumagai et al.~\cite {kumagai1968stress} for example.

\section{Method}

\subsection{Circular Hole Stress Raiser}

To investigate the properties of a circular hole stress raiser, strain gauges
were used to measure the stress at several locations of interest in a plate 
with a circular hole placed under axial load. Figure~\vref{fig:experiment-dims} 
details the dimensions of the plate used in the experiment.

\begin{figure}[h]
    \centering
    \def\svgwidth{0.48\textwidth}
    \input{img/experiment_dims.pdf_tex}
    \caption{Dimensions of plate with circular hole.}
    \label{fig:experiment-dims}
\end{figure}

The apparatus included the plate, fixed at one end with a thumb screw at the 
other to allow a range of loads to be applied. Strain gauges at the 
locations labelled in Figure~\vref{fig:experiment-gauges} were connected to a 
\textsc{PC} such that the strain could be measured.

\begin{figure}[h]
    \centering
    \def\svgwidth{0.48\textwidth}
    \input{img/experiment_gauges.pdf_tex}
    \caption{Positions of strain gauges on plate.}
    \label{fig:experiment-gauges}
\end{figure}

Because gauge 2 was located in the centre of the plate away from the
stress raiser, the strain measured at gauge 2 was taken as the nominal strain
and the independent variable. The dependent variables were the strains at gauges 1 and 3--9.

Three repeat readings were taken for five different nominal strains: 0, 
\num{200e-6}, \num{400e-6}, \num{600e-6} and \num{800e-6}. This avoided
exceeding the \num{1000e-6} limit for the apparatus, and gave sufficient repeat
readings to determine errors.

Concept Analyst\textsuperscript{\textregistered} \FEA software was used to
simulate the local stress concentrations in the plate to supplement the
experimental results.

\subsection{Shoulder Fillet Stress Raiser}

Concept Analyst \FEA software was also used to investigate how changing the
geometry of a shoulder fillet stress raiser affected the stress concentration
factor. Figure~\vref{fig:simulation-dims} details the the defining dimensions of
the feature.

\begin{figure}[h]
    \centering
    \def\svgwidth{0.48\textwidth}
    \input{img/simulation_dims.pdf_tex}
    \caption{Dimensions of plate with shoulder fillets.}
    \label{fig:simulation-dims}
\end{figure}

In practice, the plate detailed in Figure~\ref{fig:simulation-dims} was modelled
with the feature height $d$ kept constant at 100~\mm. To investigate the effects
of varying the ratio $H/d$, the simulations were run with $H$ at 110~\mm,
130~\mm and 200~\mm. For each $H$ value, $r/d$ was also varied by running the
simulations with six different feature radii in the range 5~\mm--10~\mm.

$L$ was chosen to be ten times greater than $H$, sufficiently long such that the
plate behaved as if the far-field stress was applied an infinite distance away
from the shoulder fillets. According to Kumagai et al. \cite{kumagai1968stress},
\eqref{eq:shoulder-conc} applies when $L/H > 5$.

The nominal far-field stress $S$ was applied to the plate using a traction of
magnitude 100~\MPa. The exact magnitude of this load was not important as stress
concentration factors were later normalised by the chosen far-field stress to
find the stress concentration factors.

\section{Discussion}

\subsection{Circular Hole Stress Raiser}

Figure~\vref{fig:experiment-results}


\begin{figure*}[t]
    \centering
    \begin{footnotesize}
        Error bars: $\pm 1.96$ \textsc{SE}.
    \end{footnotesize}
    \caption{Strains plotted against the nominal strain.}
    \label{fig:experiment-results}
\end{figure*}

\begin{table*}[t]
    \centering
    \caption{Strain concentration measured in plate with circular hole at
        different strains.}
    \label{tab:strain-conc}
    \begin{threeparttable}
        \begin{tabular}{
            @{}
            S[separate-uncertainty=true,
              table-format=3.1,
              table-figures-uncertainty=1]
            @{\phantom{$-$}}
            *{5}{        
                S[separate-uncertainty=true,
                  table-format=-1.3,
                  table-figures-uncertainty=1]
                @{}
            }
        }
            \toprule
            {$\epsilon_{max},\;\scriptstyle10^{-6}$} &
            {$K_{\epsilon_1}$} &
            {$K_{\epsilon_2}$} &
            {$K_{\epsilon_3}$} &
            {$K_{\epsilon_4}$} &
            {$K_{\epsilon_5}$} \\
            \cmidrule(r){1-1}\cmidrule{2-6}
            1.3\pm3.2 &
            {\phantom{.00}$-10\pm490$\phantom{.0}} &
            {\phantom{$-$.000}$1\pm0$\phantom{.000}} &
            {\phantom{$-$.000}$0\pm130$\phantom{.0}} &
            {\phantom{$-$.00}$10\pm490$\phantom{.0}} &
            {\phantom{$-$.000}$0\pm1000$\phantom{.}} \\
            198.8\pm5.9 & -0.293\pm0.059 & {\phantom{$-$.000}$1\pm0$\phantom{.000}} & 0.348\pm0.020 & 1.059\pm0.069 & 1.120\pm0.060 \\
            401.8\pm3.5 & -0.310\pm0.016 & {\phantom{$-$.000}$1\pm0$\phantom{.000}} & 0.365\pm0.032 & 1.102\pm0.066 & 1.073\pm0.079 \\
            598.5\pm1.8 & -0.320\pm0.007 & {\phantom{$-$.000}$1\pm0$\phantom{.000}} & 0.352\pm0.018 & 1.050\pm0.029 & 1.051\pm0.028 \\
            803.0\pm3.5 & -0.303\pm0.004 & {\phantom{$-$.000}$1\pm0$\phantom{.000}} & 0.349\pm0.013 & 1.037\pm0.022 & 1.015\pm0.020 \\
            \cmidrule{2-6}
            {\bf Mean\tnote{$\dagger$}}
                        & -0.307\pm0.003 & {\phantom{$-$.000}$1\pm0$\phantom{.000}} & 0.351\pm0.009 & 1.046\pm0.016 & 1.035\pm0.015 \\
            \bottomrule
            {$\epsilon_{max},\;\scriptstyle10^{-6}$} &
            {$K_{\epsilon_6}$} &
            {$K_{\epsilon_7}$} &
            {$K_{\epsilon_8}$} &
            {$K_{\epsilon_9}$} \\
            \cmidrule(r){1-1}\cmidrule{2-5}
            1.3\pm3.2 & 
            {\phantom{.00}$-10\pm490$\phantom{.0}} &
            {\phantom{.00}$-10\pm490$\phantom{.0}} &
            {\phantom{$-$.000}$0\pm260$\phantom{.0}} & 
            {\phantom{.00}$-10\pm490$\phantom{.0}} \\
            198.8\pm5.9 & -1.603\pm0.141 & 4.539\pm0.276 & 1.837\pm0.141 & 0.800\pm0.067 \\
            401.8\pm3.5 & -1.582\pm0.083 & 4.645\pm0.233 & 1.804\pm0.108 & 0.809\pm0.048 \\
            598.5\pm1.8 & -1.568\pm0.034 & 4.587\pm0.082 & 1.743\pm0.049 & 0.768\pm0.021 \\
            803.0\pm3.5 & -1.558\pm0.035 & 4.570\pm0.063 & 1.718\pm0.049 & 0.770\pm0.025 \\
            \cmidrule{2-5}
            {\bf Mean\tnote{$\dagger$}}
                        & -1.566\pm0.023 & 4.578\pm0.048 & 1.743\pm0.032 & 0.774\pm0.015 \\
            \cmidrule[\heavyrulewidth]{1-5}
        \end{tabular}
        \begin{tablenotes}
            \footnotesize   
            \item Values: $\text{mean}\pm1.96\;\text{\SE}$, sample size: 3.
            \item[$\dagger$] Maximum likelihood weighted average using weight 
                $\text{\SE}^{-2}$.
        \end{tablenotes}
    \end{threeparttable}
\end{table*}


\begin{table}[H]
    \centering
    \begin{tabular}{
        c
        S[separate-uncertainty=true,
          table-format=+1.3,
          table-figures-uncertainty=1]
    }
        $K_{\sigma_5}=$ &  1.035\pm0.015 \\
        $K_{\sigma_6}=$ & -1.566\pm0.023\\
        $K_{\sigma_7}=$ &  4.578\pm0.048  \\
        $K_{\sigma_9}=$ &  0.774\pm0.015 \\
    \end{tabular}
\end{table}


\subsection{Shoulder Fillet Stress Raiser}

\section{Conclusions}

% References.
\printbibliography{}

\end{document}