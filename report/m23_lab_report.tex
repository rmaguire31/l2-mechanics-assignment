\documentclass[a4paper,11pt,twocolumn]{article}
\usepackage[latin1]{inputenc}
\usepackage[english]{babel}
\usepackage{amsmath}
\usepackage{amsfonts}
\usepackage{amssymb}
\usepackage{mathtools}
\usepackage{cancel}

\usepackage{titling}
\usepackage{nomencl}
\usepackage{siunitx}
\usepackage[style=ieee,backend=bibtex]{biblatex}
\usepackage[font={small}]{caption}

\usepackage{graphicx}
\usepackage{color}

\usepackage{booktabs}
\usepackage{threeparttable}
\usepackage{calc}
\usepackage{fancyhdr}
\usepackage{float}

\usepackage{textcomp}
\usepackage{varioref}
\usepackage{xspace}
\usepackage[activate={true,nocompatibility},final,tracking=true,kerning=true,spacing=nonfrench,factor=1100,stretch=10,shrink=10]{microtype}
% activate={true,nocompatibility} - activate protrusion and expansion final -
% enable microtype; use "draft" to disable tracking=true, kerning=true,
% spacing=true - activate these techniques factor=1100 - add 10% to the
% protrusion amount (default is 1000) stretch=10, shrink=10 - reduce
% stretchability/shrinkability (default is 20/20) Reduce tracking around small
% caps to 40%
\SetTracking{encoding={*}, shape=sc}{40}

% Document info.
\author{Z0966990}
\title{M23 Lab Report}
\date{\today}

% Path to images.
\graphicspath{{img/}}

% Setup nomenclature.
\newlength{\nomtitlesep}
\setlength{\nomtitlesep}{\nomitemsep}
\setlength{\nomitemsep}{-0.5\parsep}
\renewcommand\nomgroup[1]{
    \ifthenelse{\equal{#1}{A}}{
        \itemsep\nomtitlesep
        \item[\textbf{Acronyms}]
        \itemsep\nomitemsep}{
    \ifthenelse{\equal{#1}{B}}{
        \itemsep\nomtitlesep
        \item[\textbf{Background}]
        \itemsep\nomitemsep}{
    \ifthenelse{\equal{#1}{C}}{
        \itemsep\nomtitlesep
        \item[\textbf{Experiment}]
        \itemsep\nomitemsep}{
    %else
        \itemsep\nomtitlesep
        \item[\textbf{\textcolor{red}{Undefined}}]
        \itemsep\nomitemsep
}}}}
\makenomenclature

% Setup bibiliography.
\addbibresource{m23_lab_report.bib}

% Header and footer.
\pagestyle{fancy}
\fancyhf{}
\lhead{\thetitle}
\rhead{\theauthor}
\cfoot{\thepage}
\renewcommand{\headrulewidth}{0pt}
\renewcommand{\footrulewidth}{0pt}

% Macros.
\newcommand{\CSA}{\textsc{CSA}\xspace}
\newcommand{\BEM}{\textsc{BEM}\xspace}
\newcommand{\SE}{\textsc{SE}\xspace}

\newcommand{\GPa}{\si{\giga\pascal}\xspace}
\newcommand{\MPa}{\si{\mega\pascal}\xspace}
\newcommand{\mm}{\si{\milli\meter}\xspace}

\begin{document}
    
% Title page.
\begin{titlepage}
    \centering
    \vspace*{\fill}
    \includegraphics[width=0.5\textwidth]{Durham.png}\\
    \vspace*{\fill}
    \LARGE\thetitle\\
    \large\theauthor\\
    \large L2 Mechanical Engineering\\
    \large\thedate\\
    \vspace*{\fill}
\end{titlepage}

% Main matter.
\renewcommand{\abstractname}{\large Abstract}
\twocolumn[
\begin{@twocolumnfalse}
    \begin{abstract}
        In this report it is detailed how experimentation, computational
        simulation and analytical approaches were used to investigate how stress
        and strain concentrations in plates vary around features.
    \end{abstract}
\end{@twocolumnfalse}
\vspace{\parsep}
]

% Acronyms
\nomenclature[A0]{\CSA}{Computational stress analysis}
\nomenclature[A1]{\BEM}{Boundary element method}
\nomenclature[A2]{\SE}{Standard error}
\nomenclature[AA]{~}{\vspace*{-\baselineskip}}

% Theory
\nomenclature[B00]{$E$}{Young's modulus of elasticity
    \hspace*{\fill}\GPa}
\nomenclature[B01]{$H$}{Height of plate
    \hspace*{\fill}\mm}
\nomenclature[B02]{$K_\epsilon$}{True strain concentration factor
    \hspace*{\fill}---}
% \nomenclature[B03]{$K_{\epsilon_t}$}{Theoretical strain concentration factor\\
%     \hspace*{\fill}---}
\nomenclature[B04]{$K_\sigma$}{True stress concentration factor
    \hspace*{\fill}---}
\nomenclature[B05]{$K_{\sigma_t}$}{Theoretical stress concentration factor\\
    \hspace*{\fill}---}
\nomenclature[B06]{$L$}{Length of plate
    \hspace*{\fill}\mm}
\nomenclature[B07]{$S$}{Far-field stress
    \hspace*{\fill}\MPa}
\nomenclature[B08]{$d$}{Height/diameter of plate feature
    \hspace*{\fill}\mm}
\nomenclature[B09]{$r$}{Radius of plate feature
    \hspace*{\fill}\mm}
% \nomenclature[B10]{$\gamma_{uv}$}{Shear strain in $uv$-plane
%     \hspace*{\fill}---}
\nomenclature[B11]{$\epsilon_{nom}$}{Nominal strain
    \hspace*{\fill}---}
\nomenclature[B12]{$\epsilon_{ww}$}{Normal strain in $w$-direction
    \hspace*{\fill}---}
\nomenclature[B13]{$\nu$}{Poisson's ratio
    \hspace*{\fill}---}
\nomenclature[B14]{$\sigma_{max}$}{Maximum plate stress
    \hspace*{\fill}\MPa}
\nomenclature[B15]{$\sigma_{nom}$}{Nominal stress
    \hspace*{\fill}\MPa}
\nomenclature[B16]{$\sigma_{ww}$}{Normal stress in $w$-direction
    \hspace*{\fill}\MPa}
% \nomenclature[B17]{$\sigma_{uv}$}{Shear stress in $uv$-plane
%     \hspace*{\fill}\MPa}
\nomenclature[BB]{}{\vspace{-\baselineskip}}

% Experiment
\nomenclature[C0]{$K_{\epsilon_i}$}{Strain concentration factor at gauge $i$\\
    \hspace*{\fill}---}
\nomenclature[C1]{$K_{\sigma_i}$}{Stress concentration factor at gauge $i$\\
    \hspace*{\fill}---}
\nomenclature[C2]{$\epsilon_i$}{Strain at gauge $i$
    \hspace*{\fill}---}
\nomenclature[C3]{$\sigma_i$}{Stress at gauge $i$
    \hspace*{\fill}\MPa}
\nomenclature[CC]{}{\vspace{-\baselineskip}}

\printnomenclature

\section{Introduction}

When designing structural components, engineers typically look to the most
stressed point in that component to determine and specify failure criteria and
factors of safety.

In simple systems, engineers can use intuition to identify candidates for
possible points of failure. In more complex systems, such as components
with smoothly varying features, more rigorous approaches to stress analysis 
are required. Experimentation, simulation and theory can all identify the most 
severely stressed point in a component.

In this report, finite width plates with either circular holes or shoulder
fillets were analysed. Circular holes are a common feature present in many
parts, allowing for fixtures such as screws and bolts to be used. Shoulder
fillets are introduced to reduce the peak stress experienced by components with
sharp internal corners.

\section{Background}

Plane stress problems are defined by their zero out of plane stress. The most 
common plane stress situations involve stresses resolved on the surface of 
plates.

On the plate boundary, the only non-zero stress component is tangential to
the plate boundary. The relationship between the tangential stress and strain 
can be described using of Hooke's law under axial load, given by 
\mbox{Hibbeler~\cite[p.~88]{hibbeler2017mechanics}}:
\begin{equation} \label{eq:hooke-axial}
    \sigma_{xx} = E\epsilon_{xx}
\end{equation}
where $x$ is the axial or tangential direction.

In the axial load case, shear strain is also zero. The Poisson
effect can be used to determine Poisson's ratio if the axial and
transverse strains are known. The following expression was also given by
\mbox{Hibbler~\cite[p.~106]{hibbeler2017mechanics}}:
\begin{equation} \label{eq:poisson}
    \nu = -\frac{\epsilon_{xx}}{\epsilon_{yy}} 
        = -\frac{\epsilon_{xx}}{\epsilon_{zz}}
\end{equation}

\subsection{Stress Concentrations}

The stress concentration factor is a dimensionless quantity which describes how
much greater the maximum principle stress in the feature is than the 
nominal---or  average---stress in the part. Similarly, there also exists a 
strain concentration factor, which is a measure of relative deformation.
According to \mbox{Pilkey~\cite[p.~4]{pilkey2008peterson}}:
\begin{align}
    \label{eq:stress-conc}
    K_\sigma &= \frac{\sigma_{max}}{\sigma_{nom}} \\
    \label{eq:strain-conc}
    K_\epsilon &= \frac{\epsilon_{max}}{\epsilon_{nom}}
\end{align}

For simple features---or stress raisers---such as the 
circular hole in Figure~\vref{fig:hole-contour}, the \emph{theoretical} stress
concentration factor $K_{\sigma_t}$ can be determined analytically as described
by \mbox{Pilkey~\cite[pp.~180--181]{pilkey2008peterson}}:
\begin{equation} \label{eq:inf-hole-conc}
    K_{\sigma_t} = 3
\end{equation}

\begin{figure}[h]
    \centering
    \def\svgwidth{0.48\textwidth}
    \input{img/hole_conc.pdf_tex}
    \caption{Circular hole stress concentrations.}
    \label{fig:hole-contour}
\end{figure}

Pilkey also offers equations to determine the local stress concentrations at
any point in the plate; used to produce the contour plot in 
Figure~\ref{fig:hole-contour}. These reveal most stressed---red---points lie
at the top and bottom of the hole.

However, when the plate has a finite width as illustrated in 
Figure~\vref{fig:hole-dims}, the problem becomes much harder to solve
analytically. Howland~\cite{howland1930stresses} derived a more generalised 
stress concentration factor expression at the top and bottom of the hole, but
not for the local stress concentrations throughout the rest of the plate:
\begin{equation} \label{eq:hole-conc}
    K_{\sigma_t} = \frac{2 + (1 - d/H)^3}{1 - d/H}
\end{equation}

\begin{figure}[h]
    \centering
    \def\svgwidth{0.48\textwidth}
    \input{img/hole_dims.pdf_tex}
    \caption{Finite width plate with circular hole.}
    \label{fig:hole-dims}
\end{figure}

Notice, in the limit as $H\rightarrow\infty$,
$\eqref{eq:hole-conc}\rightarrow\eqref{eq:inf-hole-conc}$.

For the shoulder fillets detailed in Figure~\vref{fig:shoulder-dims}, stress
concentration factors are derived empirically rather than analytically.
\mbox{Pilkey~\cite[p.~151]{pilkey2008peterson}} provides a chart relating the
ratios $H/d$ and $r/d$ to the stress concentration factor using data from
multiple sources.

\begin{figure}[h]
    \centering
    \def\svgwidth{0.48\textwidth}
    \input{img/shoulder_dims.pdf_tex}
    \caption{Finite width plate with shoulder fillets.}
    \label{fig:shoulder-dims}
\end{figure}

Therefore, to determine the location of the most stressed points, engineers must
either measure or simulate the stress concentration factors in the plate,
particularly for complex stress raisers.

\BEM is modern computational method which divides the complex part into simpler 
elements using a mesh, resolving the stresses in each element to model the 
stress field.

\section{Method}

\subsection{Circular Hole Stress Raiser}

To investigate the properties of a circular hole stress raiser, strain gauges
were used to measure the stress at several locations of interest in a plate 
with a circular hole placed under axial load. Figure~\vref{fig:experiment-dims} 
details the dimensions of the plate used in the experiment.

\begin{figure}[h]
    \centering
    \def\svgwidth{0.48\textwidth}
    \input{img/experiment_dims.pdf_tex}
    \caption{Dimensions of the experiment plate.}
    \label{fig:experiment-dims}
\end{figure}

The apparatus included the plate, fixed at one end with a thumb screw at the 
other to allow a range of loads to be applied. Strain gauges at the 
locations labelled in Figure~\vref{fig:experiment-gauges} were connected to a 
\textsc{PC} such that the strain could be measured.

\begin{figure}[h]
    \centering
    \def\svgwidth{0.48\textwidth}
    \input{img/experiment_gauges.pdf_tex}
    \caption{Positions of strain gauges on the experiment plate.}
    \label{fig:experiment-gauges}
\end{figure}

Because gauge 2 was located in the centre of the plate away from the
stress raiser, the strain measured at gauge 2 was taken as the nominal strain.
The independent variable was the maximum strain measured across all the strain
gauges, expected to be the strain measured at gauge 7 at the bottom of the
hole. The strains measured at the other gauges were the dependent variables.

Three repeat readings were taken for five different nominal strains: 0, 
\num{200e-6}, \num{400e-6}, \num{600e-6} and \num{800e-6}. This avoided
exceeding the \num{1000e-6} limit for the apparatus, and gave sufficient repeat
readings to determine errors.

Concept Analyst\textsuperscript{\textregistered} \BEM software was used to
simulate the local stress concentrations in the plate to supplement the
experimental results.

\subsection{Shoulder Fillet Stress Raiser}

Concept Analyst \BEM software was also used to investigate how changing the
geometry of a shoulder fillet stress raiser affected the stress concentration
factor. Figure~\vref{fig:simulation-dims} details the the defining dimensions of
the feature.

\begin{figure}[h]
    \centering
    \def\svgwidth{0.48\textwidth}
    \input{img/simulation_dims.pdf_tex}
    \caption{Dimensions of plate with shoulder fillets.}
    \label{fig:simulation-dims}
\end{figure}

In practice, the plate detailed in Figure~\ref{fig:simulation-dims} was modelled
with the feature height $d$ kept constant at 100~\mm. To investigate the effects
of varying the ratio $H/d$, the simulations were run with $H$ at 110~\mm,
130~\mm and 200~\mm. For each $H$ value, $r/d$ was also varied by running the
simulations with six different feature radii in the range 5~\mm--10~\mm.

$L$ was chosen to be ten times greater than $H$, operating under the assumption
that this was sufficiently long such that the plate behaved as if the far-field stress was applied an infinite distance away from the shoulder fillets.

The nominal far-field stress $S$ was applied to the plate using a traction of
magnitude 100~\MPa. The exact magnitude of this load was not important as
the stresses were later normalised by the chosen far-field stress to find the 
stress concentration factors.

\section{Results}

Figure~\vref{fig:experiment-results} details the strains measured at each gauge,
plotted against the nominal strain at gauge 2, $\epsilon_2$ or $\epsilon_{nom}$.

\begin{figure}[h]
    \centering
    \def\svgwidth{0.48\textwidth}
    \input{img/experiment_results.pdf_tex}
    \begin{footnotesize}
        Error bars: $\pm3\;\text{\SE}$.
    \end{footnotesize}
    \caption{Strains measured in the plate, plotted against nominal strain.}
    \label{fig:experiment-results}
\end{figure}

\section{Discussion}

\subsection{Circular Hole Stress Raiser}

\begin{table*}[t]
    \centering
    \caption{Local strain concentration factors measured in the experiment at
        each strain.} \label{tab:strain-conc}
    \begin{threeparttable}
        \begin{tabular}{
            @{}
            S[separate-uncertainty=true,
              table-format=3.1,
              table-figures-uncertainty=1]
            @{\phantom{$-$}}
            *{5}{        
                S[separate-uncertainty=true,
                  table-format=-1.3,
                  table-figures-uncertainty=1]
                @{}
            }
        }
            \toprule
            {$\epsilon_{max},\;\scriptstyle10^{-6}$} &
            {\phantom{$-$}$K_{\epsilon_1}$} &
            {\phantom{$-$}$K_{\epsilon_2}$} &
            {\phantom{$-$}$K_{\epsilon_3}$} &
            {\phantom{$-$}$K_{\epsilon_4}$} &
            {\phantom{$-$}$K_{\epsilon_5}$} \\
            \cmidrule(r){1-1}\cmidrule{2-6}
            1.3\pm1.7 &
            {\phantom{.00}$-10\pm250$\phantom{.0}} &
            {\phantom{$-$.000}$1\pm0$\phantom{.000}} &
            {\phantom{.000}$-3\pm68$\phantom{.00}} &
            {\phantom{$-$.00}$10\pm250$\phantom{.0}} &
            {\phantom{.00}$-30\pm530$\phantom{.0}} \\
            198.8\pm3.0 & -0.293\pm0.030 & {\phantom{$-$.000}$1\pm0$\phantom{.000}} & 0.348\pm0.010 & 1.059\pm0.035 & 1.120\pm0.031 \\
            401.8\pm1.8 & -0.310\pm0.008 & {\phantom{$-$.000}$1\pm0$\phantom{.000}} & 0.365\pm0.016 & 1.102\pm0.033 & 1.073\pm0.040 \\
            598.5\pm0.9 & -0.320\pm0.004 & {\phantom{$-$.000}$1\pm0$\phantom{.000}} & 0.352\pm0.009 & 1.050\pm0.015 & 1.051\pm0.014 \\
            803.0\pm1.8 & -0.303\pm0.002 & {\phantom{$-$.000}$1\pm0$\phantom{.000}} & 0.349\pm0.007 & 1.037\pm0.011 & 1.015\pm0.010 \\
            \cmidrule{2-6}
            {\bf Mean\tnote{$\dagger$}}
                        & -0.307\pm0.002 & {\phantom{$-$.000}$1\pm0$\phantom{.000}} & 0.351\pm0.005 & 1.046\pm0.008 & 1.035\pm0.008 \\
            \bottomrule
            {$\epsilon_{max},\;\scriptstyle10^{-6}$} &
            {\phantom{$-$}$K_{\epsilon_6}$} &
            {\phantom{$-$}$K_{\epsilon_7}$} &
            {\phantom{$-$}$K_{\epsilon_8}$} &
            {\phantom{$-$}$K_{\epsilon_9}$} \\
            \cmidrule(r){1-1}\cmidrule{2-5}
            1.3\pm1.7 & 
            {\phantom{.00}$-10\pm250$\phantom{.0}} &
            {\phantom{.00}$-10\pm250$\phantom{.0}} &
            {\phantom{$-$.000}$0\pm133$\phantom{.0}} & 
            {\phantom{.00}$-10\pm250$\phantom{.0}} \\
            198.8\pm3.0 & -1.603\pm0.072 & 4.539\pm0.141 & 1.837\pm0.072 & 0.800\pm0.034 \\
            401.8\pm1.8 & -1.582\pm0.042 & 4.645\pm0.119 & 1.804\pm0.055 & 0.809\pm0.025 \\
            598.5\pm0.9 & -1.568\pm0.017 & 4.587\pm0.042 & 1.743\pm0.025 & 0.768\pm0.011 \\
            803.0\pm1.8 & -1.558\pm0.018 & 4.570\pm0.032 & 1.718\pm0.025 & 0.770\pm0.013 \\
            \cmidrule{2-5}
            {\bf Mean\tnote{$\dagger$}}
                        & -1.566\pm0.012 & 4.578\pm0.025 & 1.743\pm0.017 & 0.774\pm0.008 \\
            \cmidrule[\heavyrulewidth]{1-5}
        \end{tabular}
        \begin{tablenotes}
            \footnotesize   
            \item Values: $\text{mean}\,\pm\,\text{\SE}$, sample size: 3.
            \item[$\dagger$] Maximum likelihood weighted average using weight 
                $\text{\SE}^{-2}$.
        \end{tablenotes}
    \end{threeparttable}
\end{table*}

Gauges 5, 6, 7 and 9 were located on the plate boundary, so the axial load
form of Hooke's law applied. Substituting \eqref{eq:hooke-axial} into 
\eqref{eq:stress-conc} yields the expression for strain concentration factor 
given in \eqref{eq:strain-conc}:
\begin{equation} \label{eq:equiv-conc}
    K_\sigma = \frac{\sigma_{max}}{\sigma_{nom}}
             = \frac{\cancel{E}\epsilon_{max}}{\cancel{E}\epsilon_{nom}}
             = K_\epsilon
\end{equation}

From \eqref{eq:equiv-conc}, it can be deduced that the local stress
concentration factors at the locations of these gauges were the same as the
local strain concentration factors calculated in Table~\ref{tab:strain-conc}.
Therefore:
\begin{align*}
        K_{\sigma_5} &= \phantom{-}1.035\pm0.008 \\
        K_{\sigma_6} &=          - 1.566\pm0.012 \\
        K_{\sigma_7} &= \phantom{-}4.578\pm0.025  \\
        K_{\sigma_9} &= \phantom{-}0.774\pm0.008
\end{align*}

Looking at the contour plot from the simulation in 
Figure~\ref{fig:experiment-simulation} reveals the maximum stress coincided
with the location of gauge 7. $K_{\sigma_7}$ is an empirical
measurement of the overall stress concentration factor in the plate.

\begin{figure}[h]
    \centering
    \caption{Contour}
    \label{fig:experiment-simulation}
\end{figure}

The maximum stress recorded in Figure~\ref{fig:experiment-simulation} was
4.306~\MPa and the far-field stress applied was 100~\MPa. Applying 
\eqref{eq:stress-conc} yields a prediction of the stress concentration factor.
Another prediction was obtained by plugging the dimensions of the plate in
Figure~\ref{fig:experiment-dims} into \eqref{eq:hole-conc}. These are compared
to the empirical value obtained:
\begin{align*}
    \text{Analytical prediction} && K_{\sigma_t} &= 4.250 \\
    \text{Simulation prediction} && K_{\sigma_t} &= 4.306 \\
           \text{Value obtained} && K_{\sigma_7} &= 4.578\pm0.012
\end{align*}

\subsection{Shoulder Fillet Stress Raiser}

\section{Conclusions}

% References.
\printbibliography{}

\end{document}